\chapter{Simon's algorithm}
\section{Introduction}
Let $N = 2^n$ and the set $\{0, 1, .. N - 1\}$ be written as $\{0, 1\}^n$ and let $i + j$ be bitwise modulo 2 addition of $i$ and $j$. \\
For example, $110 + 011$ would result in $101$ as $1 + 0 = 1\mod2$ for the first bit, $1 + 1 = 0\mod2$ for the second bit and $0 + 1 = 1\mod2$ for the third bit. 
\section{The problem}
For a given $N$, let there be a secret function $x = (x_{0}, x_{1},..., x_{N - 1})$ where $x_{i} = \{0, 1\}^n$ with the property that for some unknown $s$, $x_{i} = x_{j}$ iff $i = j$ or $i = j + s$. \\ \\
Our task is to find out $s$ using the function $x$
\section{Classical solutions}
One of the most simple and efficient classical algorithms for this is to randomly get $X$ values from the function $x$. Let the set of distinct numbers be $\{a_{1}, a_{2},...,a_{X}\}$. The probability that atleast 2 of them yield us the same value can be calculated with the help of the birthday paradox. \\
The number of pairs in our selected set that could yield us the same value are $\Comb{X}{2}$ $\sim \frac{X^2}{2}$ and the probability for any given pair to have the same value is $\frac{1}{N - 1}$. \\
Using linearity of expectation, we can expect approximately $\frac{X^2}{2 * N}$ pairs to have the same value. \\
Thus, choosing $T = \sqrt{2^\left(n + 1\right)}$, we can expect about $1$ pair to have the same value and thus find the value of $s$.
\section{Quantum solution}
Given the exponential complexity for the simon's algorithm using classical computers, using the quantum algorithms we can improve the time complexity to linear time. \\
\begin{enumerate}
    \item We use 2n zero qubits $\ket{0^n}\ket{0^n}$ and then apply hadamard transforms to the first n qubits only, which will result in state:
    \begin{equation*}
        \frac{1}{\sqrt{2^n}} \sum_{i \in \{0,1\}^n} \ket{i}\ket{0^n}
    \end{equation*}
    \item Now this state is passed through the black box, that is given to us and is known to be periodic, this query turns this into:
    \begin{equation*}
        \frac{1}{\sqrt{2^n}} \sum_{i \in \{0,1\}^n} \ket{i}\ket{x_i}
    \end{equation*}
    \item Now the second n-bit register is measured and the state collapses to one of the given distinct answers that are possible for the function with probability $\frac{1}{\sqrt{2^n}}$, now the state can be written as:
    \begin{equation*}
        \frac{1}{\sqrt{2}}(\ket{i} + \ket{i \bigoplus s} \ket{x_i})
    \end{equation*}
    \item Now we will ignore the second n bit register that have the $\ket{x_i}$ state and then apply Hadamard transforms(inverse fourier transform) to the first n qubits. We can write the state further as:
    \begin{equation*}
        \frac{1}{\sqrt{2^{n+1}}} \Bigg(\sum_{j\in\{0,1\}^n}(-1)^{i.j}\ket{j}+\sum_{j\in\{0,1\}^n}(-1)^{(i\bigoplus s).j}\ket{j} \Bigg)
    \end{equation*}
    \item Now this can be written as(using $(i\bigoplus s).j = (i.j)\bigoplus (s.j)$:
    \begin{equation*}
        \frac{1}{\sqrt{2^{n+1}}} \Bigg(\sum_{j\in\{0,1\}^n}(-1)^{i.j}(1+(-1)^{(s.j)}\ket{j}) \Bigg)
    \end{equation*}
    \item Now if we look at the state equation we can see that $\ket{j}$ has non-zero amplitude if and only if $s.j=0 \mod 2$. Measuring the state gives us a uniformly random element from the set $\{j | s.j = 0 \mod 2\}$. So we will get a linear equation that gives information about s since we already know what j is(result of measurement of after the inverse fourier transform is done).
    \item Now we repeat these steps until we get distinct $n-1$ linear equation for s. The solutions to linear equation will give us correct s(can be done in complexity of $\mathcal{O}(n^3)$. You will get $n-1$ linearly independent equations in $4n$ steps of this algorithm with $0.99\%$ probability.
\end{enumerate}

\section{Analysis and comparison}
Thus, we see an exponential speed up in this case. \\
The classical algorithm gives an expected value of $\sqrt{\frac{2^n}{2}}$ required queries to find out the value of $s$ while the quantum algorithm can find $s$ in 4n queries and some polynomial classical computations(for solving the linear equations) with $0.99$ probability. \\
This speeds up our expected query complexity from $\Omega(\sqrt{N})$ to $\mathcal{O}(n)$ \\
The speed up serves a major speed up for Shor's algorithm and is a subset of the Abelian hidden subgroup problem.
